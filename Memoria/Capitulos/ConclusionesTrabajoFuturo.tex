\chapter{Conclusiones y Trabajo Futuro}
\label{cap:conclusiones}

Dado el tiempo disponible para el desarrollo de este proyecto, no hemos podido comprobar nuestra hipótesis inicial de que en España la mayoría de canciones iban a tener un sentimiento, por lo general alegre y positivo. Al haber visto que aquellas canciones que están en inglés resultan mejor clasificadas, sabemos que tendremos que incluir un proceso de traducción al inglés o bien traducir el diccionario de sentimientos a otros idiomas. Además, cabe destacar la pérdida de contexto a la hora de analizar las palabras, ya que al tratar las palabras individualmente no podemos saber el significado total de una frase, perdiendo entonces el significado contextual de la palabra y ciñiendonos al significado común. Para ello también tendremos que implementar técnicas de procesamiento de textos mucho más avanzadas y que requieren de mucho más tiempo, esfuerzo e investigación.

Futuras aplicaciones de este trabajo de investigación pueden ser, por ejemplo buscar momentos a lo largo de la historia en los que el sentimiento de las canciones cambibie e intentar encontrar causas históricas a las que pueda estar asociado, por ejemplo guerras, crisis o cuando se gana un evento deportivo. 

Si encontrasemos alguna relación entre el sentimiento que desprende la música y eventos históricos tendríamos que plantearnos si es el arte el que imita a la vida o es la vida la que imita al arte.